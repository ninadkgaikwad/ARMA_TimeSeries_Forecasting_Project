%%Preamble

 \documentclass[12pt]{report}

\usepackage[utf8]{inputenc}
\usepackage[a4paper, width=180mm, top=25mm, bottom=25mm]{geometry}

\usepackage{graphicx}
\graphicspath{{Images/}} %% Path to find Image Files

\usepackage{amsmath}
\usepackage{amsfonts}
\usepackage{amssymb}
\numberwithin{equation}{section} % for giving numbered euations section wise otherwise use \numberwithin{equation}{subsection}

\renewcommand{\arraystretch}{1.5}

\usepackage{fancyhdr}
\pagestyle{fancy}
\fancyhead{} %% Remove all Headers
\fancyhead[RO,LE]{Development of Solar \& Wind Energy Estimation and Forecasting Application}
\fancyfoot{} %% Remove all Footers
\fancyfoot[LE,RO]{\thepage}
\fancyfoot[LO,CE]{Chapter \thechapter}
\fancyfoot[CO,RE]{Ninad K. Gaikwad}
\renewcommand{\headrulewidth}{0.4pt} %% Horizontal lines separating Header and Footers from main text
\renewcommand{\footrulewidth}{0.4pt}

\usepackage{caption} %% For using Sub-Figure and Sub-Table Environments
\usepackage{subcaption}

\usepackage{float}
\usepackage[section]{placeins} % \FloatBarrier, it avoids floating of tables and

\usepackage{natbib} %% References from Bib File

\begin{document}

%%Title Page

\begin{titlepage}

    \begin{center}
        \vspace*{1cm}
        \begin{Huge}
        \textbf{Forecasting of Solar And Wind Energy}
        \end{Huge}
                
        \vspace{1.5cm}
        \begin{Large}
        \textbf{Ninad Kiran Gaikwad}
        \end{Large}
        
       
        \vspace{2cm}
        This thesis is written in partial fulfillment of the requirements for the degree of\\
        \begin{large}
        \textit{Master of Technology}
        \end{large}
        \\
        in
        \\
        \begin{large}
        \textit{Electrical Engineering}
        \end{large}
        \\
        specializing in
        \\
        \begin{large}
        \textit{Power Electronic \& Power Systems}
        \end{large}
        
        \vspace{1.5cm}
        \includegraphics[width=0.6\textwidth]{MumU}
        
        \vspace{1.5cm}    
        Department of Electrical Engineering\\
        Sardar Patel College of Engineering\\
        University of Mumbai   
        
    \end{center}   

\end{titlepage}

%% Un-numbered Chapters

\chapter*{Abstract}
\
\
\
\
The thesis focuses on the developing an application eco-system for performing intra-hour, intra-day and day-ahead generation forecasting for Solar Photovoltaic and Wind Turbine grid connected power plants. The generation forecasting has been divided into two parts: Weather Variable Forecasting (As they are the stochastic component of the problem) and the Energy Estimation of Solar and Wind power plants (as the energy estimation becomes deterministic once the input weather variables have been forecasted). Artificial Neural Networks (ANN), Auto-Regressive Integrated Moving Average Models (ARIMA) and Weather Research and Forecasting Software (WRF) are used for intra-hour, intra-day and day-ahead generation forecasting. For running the WRF software in a distributed-memory and parallel fashion, a 4-Node Raspberry-Pi2 cluster has been developed. The entire application eco-system is developed in MATLAB using its GUI feature, which makes generation forecasting very user friendly. Six applications with their sub-modules have been designed and developed for this purpose : Data Pre-Processing Application, Solar Energy Estimation Application, Wind Energy Estimation Application, ARIMA Forecasting Application, ANN Forecasting Application, and the WRF-NETCDF Visualization \& Extraction Application. The solar energy estimation application is tested with data from Backbone 5MW and GSEC 1MW solar photovoltaic power plants, whereas the wind energy estimation application is tested with a hypothetical wind power plant (due to lack of real wind power plant data). Moreover, the ARIMA application is tested to produce intra-day generation forecasts for GSEC 1MW plant in conjugation with the solar energy estimation application. Similarly, the ANN forecasting application is test to produce intra-hour generation forecasts for GSEC 1MW plant in conjugation with the solar energy estimation application. Finally, the WRF softaware running on the developed Raspberry-Pi2 cluster is used to produce day-ahead generation forecasts for GSEC 1MW plant in conjugation the wrf-netcdf visualization \& extraction application and solar energy estimation application. All the results are well documented and presented in the thesis. The application GUIs are illustrated in the appendices.


\chapter*{Acknowledgements}
\
\
\
\
Firstly, I would like to express my sincere gratitude to my advisor Assist \textbf{Prof. Nitin G. Bhitre} for the continuous support of my Master’s study and related research, for his patience, motivation, and immense knowledge. His guidance helped me in all the time of research and writing of this thesis. I could not have imagined having a better advisor and mentor for my Master’s study.\\

Besides my advisor, I would like to thank the rest of Electrical Engineering Department: \textbf{Assist Prof. B. B. Pimple} , for their insightful comments and encouragement, but also for the hard question which helped me widen my research from various perspectives.\\

I am also thankful to \textbf{Mr. Saurabh Gavali} (Intern – GERMI) for helping me creating the base version of the Wind Energy Estimation Application. Moreover, I am thankful to \textbf{Mr. Shantanu Vaishnav} (Intern - GERMI) for helping me in ANN forecasting module.

My sincere thanks also goes to\textbf{ Dr. Sagarkumar Agravat} and \textbf{Dr. T. Harynarayana} of GERMI, who provided me an opportunity to join their Renewable Energy Research Wing as a research intern, and who gave access to the laboratory and research facilities. Without their precious support, guidance and continuous motivation it would not be possible to conduct this research.


\chapter*{Dedication}
\
\
\
\
I dedicate my dissertation work to my family and many friends. A special feeling of gratitude to my loving parents, \textbf{Dr. Kiran} and\textbf{ Dr. Madhuri Gaikwad }whose words of encouragement and push for tenacity ring in my ears. My brothers\textbf{ Varun} and \textbf{Dr. Vrushabh Gavali}; my friends \textbf{Jubin} and \textbf{Yogesh}, who have never left my side and are very special. I also dedicate this dissertation to my dear friend \textbf{Sarita Panke} who has supported me throughout the process. I will always appreciate all they have done. All of you have been like glittering stars in the long-night’s sky assuring of the bright sunrise, and I am pleased to say the dawn is nearer by an hour.

\chapter*{Epigraph}

\begin{flushright}
\begin{Large}PLATO\end{Large}\\
\vspace{5mm}
\textit{“The beginning is the most important part of the work.”}\\

\vspace{2cm}

\begin{Large}Sir Isaac Newton\end{Large}\\
\vspace{5mm}
\textit{“What certainty can there be in a Philosophy which consists in as many Hypotheses as there are Phenomena to be explained. To explain all nature is too difficult a task for any one man or even for any one age. 'Tis much better to do a little with certainty, and leave the rest for others that come after you, than to explain all things by conjecture without making sure of anything.”}\\

\vspace{2cm}

\begin{Large}Albert Einstein\end{Large}\\
\vspace{5mm}
\textit{" ... as far as the propositions of mathematics refer to reality, they are not certain; and as far as they are certain, they do not refer to reality."}\\

\vspace{2cm}

\begin{Large}Richard Feynman\end{Large}\\
\vspace{5mm}
\textit{“I have approximate answers and possible beliefs in different degrees of certainty about different things, but I'm not absolutely sure of anything.”}
\end{flushright}


%% Table of Contents

\tableofcontents
\listoffigures
\listoftables

%% Beginning of Main Chapters, these are linked to the Main file using input command

\chapter{Introduction}
\input{Chapters/Introduction}

\chapter{Data of Gujarat Solar PV Plants}
%% Chapter 2 : Data of Gujarat Solar PV Plants

\section{Intro}
\
\
\
\
Historical solar generation data is the most important resource along with historical weather data to model highly accurate prediction systems for Solar PV generation. The data used in this thesis is obtained from the Gujarat State Load Dispatch Centre (SLDC). Some analysis has been done on the raw data from SLDC to convert it into valuable information which can be used to great effect in the thesis.

\section{Solar Generation in Gujarat}
\
\
\
\
The following table lists the Solar PV plants operational in Gujarat as of August, 2015. The Table \ref{tabc2h1} gives information about the name of the plant,  the latitudes and longitudes and commissioned capacity. The generation data from 2012 to August of 2015 has been procured from the SLDC
\\

\begin{table}[H]
  \centering
%%From Excet to LATEX Add-In  
   \caption{List of Solar PV Plants in Gujarat}

    \begin{tabular}{|l|c|c|c|}
    \hline
    \textbf{Company Name} & \textbf{Lat } & \textbf{Long } & \textbf{Commissioned, MW} \bigstrut\\
    \hline
    Sunkon Energy Pvt. Ltd. & \textbf{20.9} & \textbf{71.3} & \textbf{10} \bigstrut\\
    \hline
    ACME Solar Technologies (Gujarat) Pvt. Ltd. & \textbf{22.3} & \textbf{72.4} & 15 \bigstrut\\
    \hline
    Precious Energy Services Pvt. Ltd. & \textbf{23.9} & \textbf{71.9} & \textbf{15.2} \bigstrut\\
    \hline
    Sandland Real Estate Pvt. Ltd. & \textbf{24.5} & \textbf{72.2} & \textbf{25} \bigstrut\\
    \hline
    Solitaire Energies Pvt. Ltd. & \textbf{23.9} & \textbf{71.9} & \textbf{15.01} \bigstrut\\
    \hline
    TATA Power Renewable Energy Ltd. & \textbf{22.4} & \textbf{69.0} & \textbf{25} \bigstrut\\
    \hline
    Mono Steel (India) Ltd. & \textbf{22.8} & \textbf{71.0} & \textbf{10} \bigstrut\\
    \hline
    Adani Enterprises Ltd. & \textbf{23.3} & \textbf{69.0} & 40.11 \bigstrut\\
    \hline
    Backbone Enterprises Ltd. & \textbf{23.4} & \textbf{70.6} & 5 \bigstrut\\
    \hline
    Euro Solar Power Pvt Ltd & \textbf{23.4} & \textbf{70.6} & \textbf{5.12} \bigstrut\\
    \hline
    Gujarat Mineral Development Company Ltd. & \textbf{23.8} & \textbf{68.8} & \textbf{5} \bigstrut\\
    \hline
    Integrated Coal Mining Ltd. & \textbf{23.4} & \textbf{70.7} & \textbf{9} \bigstrut\\
    \hline
    Konark Gujarat PV Pvt. Ltd. & \textbf{23.4} & \textbf{70.6} & \textbf{5} \bigstrut\\
    \hline
    Solar Semiconductor Power Company ( India) Pvt Ltd & \textbf{23.4} & \textbf{70.6} & \textbf{20} \bigstrut\\
    \hline
    Sunborne Energy Gujarat One Pvt.Ltd. & \textbf{23.4} & \textbf{70.4} & \textbf{15} \bigstrut\\
    \hline
    Unity Power Private Ltd.  & \textbf{23.4} & \textbf{70.1} & \textbf{5} \bigstrut\\
    \hline
    Welspun Urja Gujarat Pvt. Ltd. & \textbf{23.4} & \textbf{70.1} & \textbf{15.01} \bigstrut\\
    \hline
    AES Solar Energy Gujarat Pvt. Ltd. & \textbf{23.9} & \textbf{71.2} & 14.92 \bigstrut\\
    \hline
    Alex Astral Power Pvt. ltd. & \textbf{23.9} & \textbf{71.2} & 25.07 \bigstrut\\
    \hline
    Astonfield Solar (Gujarat) Private Limited  & \textbf{23.7} & \textbf{71.7} & 11.51 \bigstrut\\
    \hline
    Avtar Solar Power Pvt Ltd & \textbf{23.9} & \textbf{71.2} & 4.98 \bigstrut\\
    \hline
    Emami Cement Ltd. & \textbf{23.9} & \textbf{71.2} & \textbf{10.06} \bigstrut\\
    \hline
    GMR Gujarat Solar Power Pvt. Ltd. & \textbf{23.9} & \textbf{71.2} & \textbf{25} \bigstrut\\
    \hline
    GSPC Pipavav Power Company Limited & \textbf{23.9} & \textbf{71.2} & \textbf{5} \bigstrut\\
    \hline
    Jaihind Projects Ltd. & \textbf{23.9} & \textbf{71.2} & \textbf{5} \bigstrut\\
    \hline
    Kindle Engg \& Const Pvt Ltd & \textbf{23.9} & \textbf{71.2} & \textbf{50} \bigstrut\\
    \hline
    Lanco Infratech Ltd & \textbf{23.9} & \textbf{71.2} & \textbf{15.01} \bigstrut\\
    \hline
    Lanco Infratech Ltd (BHRD) & \textbf{23.7} & \textbf{71.6} & \textbf{5} \bigstrut\\
    \hline
    Lanco Infratech Ltd (Chandiyana) & \textbf{23.7} & \textbf{71.6} & \textbf{15.01} \bigstrut\\
    \hline
    NKG Infrastructure Ltd. & \textbf{23.9} & \textbf{71.2} & \textbf{10} \bigstrut\\
    \hline
    \end{tabular}%

%% All above Code is copied from Excel to LATEX Add-In 
    %\caption{Add caption}
    \label{tabc2h1}%

\end{table}

\begin{table}[H]
  \centering
%%From Excet to LATEX Add-In  
    \begin{tabular}{|l|c|c|c|}
    \hline
    \textbf{Company Name} & \textbf{Lat (N)} & \textbf{Long (E)} & \textbf{Commissioned, MW} \bigstrut\\
    \hline
    Palace Solar Energy Pvt. Ltd. & \textbf{23.9} & \textbf{71.2} & \textbf{15} \bigstrut\\
    \hline
    PLG Photovoltaics Ltd & \textbf{23.9} & \textbf{71.5} & \textbf{20} \bigstrut\\
    \hline
    Roha Dyechem Pvt. Ltd. & \textbf{23.9} & \textbf{71.2} & \textbf{25.04} \bigstrut\\
    \hline
    Solarfield Energy Private Limited & \textbf{23.8} & \textbf{71.2} & \textbf{20.06} \bigstrut\\
    \hline
    Sun Clean Renewable Power Pvt. Ltd. & \textbf{23.9} & \textbf{71.2} & \textbf{6} \bigstrut\\
    \hline
    Surana Telecom \& Power Ltd. & \textbf{23.9} & \textbf{71.2} & \textbf{5} \bigstrut\\
    \hline
    Torrent Solargen Limited & \textbf{23.9} & \textbf{71.2} & \textbf{51} \bigstrut\\
    \hline
    Yantra eSolar India Pvt. Ltd. & \textbf{23.9} & \textbf{71.2} & \textbf{4.95} \bigstrut\\
    \hline
    Gujarat Power Corporation Ltd. & \textbf{23.9} & \textbf{71.2} & \textbf{5} \bigstrut\\
    \hline
    SEI Solar Power Gujarat Pvt. Ltd. & \textbf{23.9} & \textbf{71.2} & \textbf{25.01} \bigstrut\\
    \hline
    ZF Steering Gear(India) Pvt. Ltd. & \textbf{23.9} & \textbf{71.2} & \textbf{5} \bigstrut\\
    \hline
    GHI Energy Pvt. Ltd. (SPV of Refex) & \textbf{21.6} & \textbf{69.9} & \textbf{10} \bigstrut\\
    \hline
    Hiraco Renewable Energy Pvt. Ltd. & \textbf{21.6} & \textbf{69.9} & \textbf{20.11} \bigstrut\\
    \hline
    Moserbaer Energy \& Development Ltd. & \textbf{21.6} & \textbf{69.9} & \textbf{15.02} \bigstrut\\
    \hline
    APCA Power Pvt. Ltd. & \textbf{21.8} & \textbf{70.1} & 5 \bigstrut\\
    \hline
    Aravali Infrapower Ltd. & \textbf{21.8} & \textbf{70.1} & 5 \bigstrut\\
    \hline
    CBC Solar Technologies Pvt. Ltd.  & \textbf{21.7} & \textbf{70.1} & 10 \bigstrut\\
    \hline
    Ganeshvani Merchandise Pvt Ltd & \textbf{21.7} & \textbf{70.1} & \textbf{5.04} \bigstrut\\
    \hline
    Ganges Green Energy Pvt Ltd. & \textbf{21.7} & \textbf{70.1} & \textbf{25.08} \bigstrut\\
    \hline
    Green Infra Solar Energy Ltd. & \textbf{21.7} & \textbf{70.1} & \textbf{10} \bigstrut\\
    \hline
    Taxus Infrastructure \& Power Project Pvt.Ltd & \textbf{23.3} & \textbf{70.0} & \textbf{5} \bigstrut\\
    \hline
    Aatash Power Pvt. Ltd. & \textbf{23.6} & \textbf{73.3} & 4.99 \bigstrut\\
    \hline
    Azure Power (Haryana) Pvt. Ltd. & \textbf{23.4} & \textbf{73.2} & 10.21 \bigstrut\\
    \hline
    Gujarat Industries Power Company Ltd. & \textbf{21.4} & \textbf{73.1} & \textbf{5.01} \bigstrut\\
    \hline
    Azure Power (Gujarat) Pvt. Ltd. & \textbf{23.2} & \textbf{71.4} & 5 \bigstrut\\
    \hline
    Chattel Constructions Private Ltd. & \textbf{23.3} & \textbf{71.8} & 25.04 \bigstrut\\
    \hline
    Dreisatz MySolar24 (P) Ltd. & \textbf{23.4} & \textbf{71.6} & 14.99 \bigstrut\\
    \hline
    EMCO Ltd. & \textbf{23.4} & \textbf{71.6} & \textbf{5} \bigstrut\\
    \hline
    ESP Urja Pvt. Ltd. & \textbf{23.5} & \textbf{71.7} & \textbf{5} \bigstrut\\
    \hline
    Louroux Bio Energies Ltd.  & \textbf{22.7} & \textbf{71.4} & \textbf{25} \bigstrut\\
    \hline
    \end{tabular}%

%% All above Code is copied from Excel to LATEX Add-In 
    %\caption{Add caption}
    %\label{}%

\end{table}

\begin{table}[H]
  \centering
%%From Excet to LATEX Add-In  
    \begin{tabular}{|l|c|c|c|}
    \hline
    \textbf{Company Name} & \textbf{Lat (N)} & \textbf{Long (E)} & \textbf{Commissioned, MW} \bigstrut\\
    \hline
    MI MySolar24 (P) Ltd,  & \textbf{23.4} & \textbf{71.6} & \textbf{14.99} \bigstrut\\
    \hline
    Millennium Synergy (Gujarat) Pvt. Ltd. & \textbf{23.4} & \textbf{71.7} & \textbf{9.27} \bigstrut\\
    \hline
    Responsive sutip ltd. & \textbf{23.1} & \textbf{71.9} & \textbf{25.06} \bigstrut\\
    \hline
    S J Green Park Energy Pvt. Ltd & \textbf{22.7} & \textbf{71.4} & \textbf{5.12} \bigstrut\\
    \hline
    Ujjawala Power Pvt Ltd & \textbf{23.1} & \textbf{71.9} & \textbf{23.06} \bigstrut\\
    \hline
    Visual Percept Solar Projects Pvt. Ltd. & \textbf{23.5} & \textbf{71.6} & \textbf{25} \bigstrut\\
    \hline
    Waa Solar Pvt. Ltd. & \textbf{22.7} & \textbf{71.4} & \textbf{10.25} \bigstrut\\
    \hline
    SSNL, SAMA & \textbf{22.3} & \textbf{73.2} & \textbf{10} \bigstrut\\
    \hline
       & \textbf{} & \textbf{Total} & \textbf{852.31} \bigstrut\\
    \hline
    \end{tabular}%


%% All above Code is copied from Excel to LATEX Add-In 
    %\caption{Add caption}
    %\label{}%

\end{table}
\\
All the above mentioned Solar Plants were located and tagged on a Google Earth File for ease of access to their geometric design for creating accurate shading analysis in PVsyst Software. The Fig \ref{figc2h1} shows a snapshot of the Google Earth File.
\\
\begin{figure}[H]
\centering
\includegraphics[scale=0.5]{GoogleEarthGuj}
\caption{Google Earth View of Solar PV Plants \geq 5MW in Gujarat}
\label{figc2h1} %% to refer use, \ref{}
\end{figure}
\\
The Fig \ref{figc2h2} gives the monthly solar energy generated by the PV plants given in Table \ref{tabc2h1} for four years from 2012 till August, 2015.
\\
\begin{figure}[H]
\centering
\includegraphics[scale=0.65]{Guj1}
\caption{Monthly Solar Power Generation in Gujarat as of Aug 2015}
\label{figc2h2} %% to refer use, \ref{}
\end{figure}
\\
The Capacity Utilization Factor (CUF)is very good measure of the efficiency of any power plant. The CUF of the entire generating capacity of PV plants in Table  \ref{tabc2h1} is calculated and presented in Fig.\ref{figc2h3}, which shows a rising efficiency of PV Power Plants in Gujarat over the course of 4 years.
\\

\begin{figure}[H]
\centering
\includegraphics[scale=0.65]{Guj2}
\caption{Capacity Utilization Factor (CUF) for entire Gujarat}
\label{figc2h3} %% to refer use, \ref{}
\end{figure}

\newpage

\section{District-Wise Solar Energy Generation in Gujarat}
\
\
\
\
The data belonging to the Solar Plants in Fig \ref{tabc2h1} is segregated district-wise, it gives some really good insights. The Table \ref{tabc2h2} contains the capacity, average yearly energy generation, average yearly CUF and average yearly MWh/MW data.
\\ 
\begin{table}[H]
  \centering
%%From Excet to LATEX Add-In  
\caption{Gujarat District-Wise Solar Energy Data}
    \begin{tabular}{|r|c|c|r|r|}
    \hline
       & \textbf{Allotted,MW} & \textbf{CUF} & \multicolumn{1}{c|}{\textbf{Energy/Year}} & \multicolumn{1}{c|}{\textbf{MWh/MW}} \bigstrut\\
    \hline
    \textbf{Amreli} & 10 & 0.174114 & 9360.5402 & 936.054 \bigstrut\\
    \hline
    \textbf{Anand} & 15 & 0.147882 & 15591.016 & 1039.401 \bigstrut\\
    \hline
    \textbf{Banaskantha} & 55 & 0.18076 & 61724.444 & 1122.263 \bigstrut\\
    \hline
    \textbf{Jamnagar} & 25 & 0.162181 & 29841.603 & 1193.664 \bigstrut\\
    \hline
    \textbf{Junagadh} & 10 & 0.173891 & 11885.173 & 1188.517 \bigstrut\\
    \hline
    \textbf{Kutch} & 129 & 0.160941 & 132019.11 & 1023.404 \bigstrut\\
    \hline
    \textbf{Patan} & 397.5 & 0.166076 & 308645.3 & 776.4662 \bigstrut\\
    \hline
    \textbf{Porbandar} & 45 & 0.150004 & 49970.088 & 1110.446 \bigstrut\\
    \hline
    \textbf{Rajkot} & 60 & 0.165073 & 57596.477 & 959.9413 \bigstrut\\
    \hline
    \textbf{Sabarkantha} & 15 & 0.162073 & 17215.268 & 1147.685 \bigstrut\\
    \hline
    \textbf{Surat} & 5  & 0.14309 & 5229.0196 & 1045.804 \bigstrut\\
    \hline
    \textbf{Surendranagar} & 195 & 0.14968 & 177200.73 & 908.7217 \bigstrut\\
    \hline
    \textbf{Vadodara} & 10 & 0.064522 & 1009.5442 & 100.9544 \bigstrut\\
    \hline
    \end{tabular}%

%% All above Code is copied from Excel to LATEX Add-In 
    %\caption{Add caption}
    \label{tabc2h2}%

\end{table}
\\
The Fig \ref{figc2h4} shows the district-wise solar energy generation in gujarat. We can see that Patan leads all the districts in energy production (due to largest number of solar plants ) whereas Vadodara has the lowest energy production.
\\

\begin{figure}[H]
\centering
\includegraphics[scale=0.65]{Dist2}
\caption{District-Wise Solar Energy Generation in Gujarat}
\label{figc2h4} %% to refer use, \ref{}
\end{figure}
\\
The Fig \ref{figc2h5} shows the district-wise MWh/MW, it is also a good indicator of the efficiency of the solar power produced. We can see that over the years these values for each state have increased and Junagadh shows the highest MWh/MW value, while Patan shows the lowest MWh/MW value.
\\

\begin{figure}[H]
\centering
\includegraphics[scale=0.65]{Dist3}
\caption{District-Wise MWh/MW in Gujarat}
\label{figc2h5} %% to refer use, \ref{}
\end{figure}
\\
The Fig \ref{figc2h6} shows the district-wise CUF in Gujarat. We can see that similar to the last graph here to Junagadh has the highest value and Patan has the lowest value. We can say that CUF and MWh/MW have a good correlation, and they show efficiency of energy generation in different ways but of similar forms.
\\

\begin{figure}[H]
\centering
\includegraphics[scale=0.65]{Dist1}
\caption{District-Wise Capacity Utilization Factor (CUF) in Gujarat}
\label{figc2h6} %% to refer use, \ref{}
\end{figure}

\section{Charanka Solar Park Analysis on the basis of PV Technology}
\
\
\
\
Charanka Solar Park in Patan district is Gujarat's largest Solar park as of August, 2015 with eighteen solar power plants and a total capacity of 272MW. Models of all the eighteen solar plants were simulated in PVsyst software and an analysis of performance of Mono-Crystalline (Mono), Poly-Crystalline (Poly) and Thin Film (Thin) PV technologies has been carried out. The Table \ref{tabc2h3} gives the results of that analysis values which have a letter P after them are simulated quantities and others are real quantities.
\\
\begin{table}[H]
  \centering
  \caption{Charaka Solar Park Analysis}
%%From Excet to LATEX Add-In  
    \begin{tabular}{|c|l|l|l|l|l|l|}
    \hline
    \textbf{} & \textbf{Name} & \textbf{MW} & \textbf{MWh} & {\textbf{MWh P}} & \textbf{MWh/MW} & \textbf{MWh/MW P} \bigstrut\\
    \hline
    \textbf{Mono} & LANCO  & 15.0 & 20506.3 & 27379.0 & 1367.1 & 1825.3 \bigstrut\\
\cline{2-7}       & \textbf{Tot Energy} & \textbf{15.0} & \textbf{20506.3} & \textbf{27379.0} &    &  \bigstrut\\
\cline{2-7}       & \textbf{MWh/MW} &    & \textbf{1367.1} & \textbf{1825.3} &    &  \bigstrut\\
\cline{2-7}       & \textbf{Average} &    &    &    & \textbf{1367.1} & \textbf{1825.3} \bigstrut\\
    \hline
    \textbf{Poly } & GSPC  & 5.0 & 8906.5 & 8358.0 & 1781.3 & 1671.6 \bigstrut\\
\cline{2-7}       & Surana  & 5.0 & 7899.3 & 9053.0 & 1579.9 & 1810.6 \bigstrut\\
\cline{2-7}       & NKG  & 10.0 & 17729.4 & 17168.0 & 1772.9 & 1716.8 \bigstrut\\
\cline{2-7}       & GMR  & 25.0 & 42603.7 & 44041.0 & 1704.1 & 1761.6 \bigstrut\\
\cline{2-7}       & Sun Edison & 25.0 & 42255.0 & 43728.0 & 1690.2 & 1749.1 \bigstrut\\
\cline{2-7}       & Emami  & 10.0 & 16958.8 & 17954.0 & 1695.9 & 1795.4 \bigstrut\\
\cline{2-7}       & GPCL & 5.0 & 7952.3 & 8863.0 & 1590.5 & 1772.6 \bigstrut\\
\cline{2-7}       & Palace & 15.0 & 26336.3 & 26946.0 & 1755.8 & 1796.4 \bigstrut\\
\cline{2-7}       & Avtar  & 5.0 & 8117.1 & 8611.0 & 1623.4 & 1722.2 \bigstrut\\
\cline{2-7}       & Torrent  & 51.0 & 77483.7 & 89184.0 & 1519.3 & 1748.7 \bigstrut\\
\cline{2-7}       & \textbf{Tot Energy} & \textbf{156.0} & \textbf{256241.9} & \textbf{273906.0} &    &  \bigstrut\\
\cline{2-7}       & \textbf{MWh/MW} &    & \textbf{1642.6} & \textbf{1755.8} &    &  \bigstrut\\
\cline{2-7}       & \textbf{Average} &    &    &    & \textbf{1671.3} & \textbf{1754.5} \bigstrut\\
    \hline
    \textbf{Thin} & Alex  & 25.0 & 41968.2 & 47372.0 & 1678.7 & 1894.9 \bigstrut\\
\cline{2-7}       & ZF  & 5.0 & 8734.6 & 8722.0 & 1746.9 & 1744.4 \bigstrut\\
\cline{2-7}       & Sun Clean & 6.0 & 10265.9 & 10408.0 & 1711.0 & 1734.7 \bigstrut\\
\cline{2-7}       & Solarified & 20.0 & 34091.4 & 36212.0 & 1704.6 & 1810.6 \bigstrut\\
\cline{2-7}       & AES  & 15.0 & 23241.9 & 26368.0 & 1549.5 & 1757.9 \bigstrut\\
\cline{2-7}       & Roha  & 25.0 & 43540.3 & 47860.0 & 1741.6 & 1914.4 \bigstrut\\
\cline{2-7}       & Yantra & 5.0 & 6981.4 & 8858.0 & 1396.3 & 1771.6 \bigstrut\\
\cline{2-7}       & \textbf{Tot Energy} & \textbf{101.0} & \textbf{168823.7} & \textbf{185800.0} &    &  \bigstrut\\
\cline{2-7}       & \textbf{MWh/MW} &    & \textbf{1671.5} & \textbf{1839.6} &    &  \bigstrut\\
\cline{2-7}       & \textbf{Average} &    &    &    & \textbf{1650.0} & \textbf{1804.1} \bigstrut\\
    \hline
    \end{tabular}%

%% All above Code is copied from Excel to LATEX Add-In 
    %\caption{Add caption}
    \label{tabc2h3}%

\end{table}
\\
The Fig \ref{figc2h7} represents the PV Technology wise solar energy generation. We can see that Poly is highest generation as most of the plants employ Poly and as only one plant uses Mono it gives the lowest generation. But, this is not an actual measurement of performance of the PV Technologies.
\\

\begin{figure}[H]
\centering
\includegraphics[scale=0.5]{Charanka2}
\caption{Charanka Solar Park PV Technology-Wise Energy Generation}
\label{figc2h7} %% to refer use, \ref{}
\end{figure}
\\
The Fig \ref{figc2h8} gives a good measure of performance of the different PV Technologies used in Charanka Solar Park, by measuring MWh/MW for all the plants using the same PV technology as a whole. We can see that with actual data Poly performs best and Mono performs the worst, however when we consider simulated values Mono performs best and Poly performs worst. 
\\

\begin{figure}[H]
\centering
\includegraphics[scale=0.5]{Charanka1}
\caption{Charanka Solar Park PV Technology-Wise MWh/MW Comparison}
\label{figc2h8} %% to refer use, \ref{}
\end{figure}
\\
The Fig \ref{figc2h9} shows the PV Technology wise CUF comparison of Charanka Solar Park. We can observe that the the real and simulated values of CUF for Poly and Thin are approximately equal validating the PVsyst models, however even in this graph the simulated values for Mono are fairly larger than the real values; which indicates some modeling error.(most probably error could be the generalised weather data which PVsyst produces) 
\\
\begin{figure}[H]
\centering
\includegraphics[scale=0.5]{Charanka3}
\caption{Charanka Solar Park PV Technology-Wise CUF Comparison}
\label{figc2h9} %% to refer use, \ref{}
\end{figure}

\section{Tracking Mechanism Performance Assessment using BACKBONE Plant Data}
\
\
\
\
The BACKBONE Enterprise LTD, 5MW Solar PV plant, in Kutch district is the only plant in whole of Gujarat to have a tracking mechanism. It uses a Single Axis (N-S axis)tracker system, details of which are given in Table \ref{tabc2h4}.
\\
\begin{table}[H]
  \centering
%%From Excet to LATEX Add-In  
    \begin{tabular}{|l|l|}
    \hline
    \textbf{Make} & SUNLINK  ViaSol Tracker \\
    \hline
    \textbf{Tracking Type (East/West)} & One-Axis Horizontal \\
    \hline
    \textbf{Tilt-Range} & +/- 45º (maximum) \\
    \hline
    \textbf{Backtracking} & Yes, Standard \\
    \hline
    \textbf{Sub-Array Rated Power} & Up to 1 MW dc \\
    \hline
    \textbf{Wind Load Capacity} & Up to 150 mph (35 mph stow) \\
    \hline
    \textbf{Time to Stow or Recover} & Less than 2 minutes \\
    \hline
    \textbf{Tracking Method} & Based on NASA time-and-location algorithm \\
    \hline
    \textbf{Drive Type} & Fluid Power \\
    \hline
    \textbf{Controller} & PLC controller utilizing industrial automation components \\
    \hline
    \textbf{Power Supply} & AC Supply from Auxiliary \\
    \hline
    \end{tabular}
%% All above Code is copied from Excel to LATEX Add-In 
    \caption{Tracker System Specifications}
    \label{tabc2h4}
\end{table}
\\
A model of the plant was created in PVsyst and simulated for Fixed Tilt (FT), Seasonal Tilt (ST), and Dual Axis Tracker (DA). We had the real values for the Single Axis Tracker (SA) and then compared them for analysis. The Table \ref{tabc2h5} shows the result of the PVsyst software simulation. 
\\

\begin{table}[H]
  \centering
     \caption{Tracker System Performance Comparison in PVsyst}
%%From Excet to LATEX Add-In  
    \begin{tabular}{|r|r|r|r|r|r|}
    \hline
    \textbf{} & \textbf{Actual} & \textbf{Fixed Tilt} & \textbf{Seasonal Tilt} & \textbf{N-S Axis} & \textbf{Dual-Axes} \\
    \hline
    \textbf{January} & 552.129 & 881.9 & 979.6 & 905   & 1174 \\
    \hline
    \textbf{February} & 642.4595 & 764.6 & 788.2 & 826   & 967 \\
    \hline
    \textbf{March} & 865.858 & 962.3 & 907.9 & 1144  & 1240 t\\
    \hline
    \textbf{April} & 916.5298 & 876   & 884   & 1143  & 1186 \\
    \hline
    \textbf{May} & 1044.885 & 863.4 & 883.2 & 1182  & 1209 \\
    \hline
    \textbf{June} & 599.0633 & 754.6 & 775.4 & 1020  & 1041 \\
    \hline
    \textbf{July} & 435.0063 & 581.8 & 595   & 730   & 740 \\
    \hline
    \textbf{August} & 425.8848 & 567.2 & 575.6 & 676   & 685 \\
    \hline
    \textbf{September} & 486.5465 & 741   & 741   & 874   & 920 \\
    \hline
    \textbf{October} & 550.0673 & 870.3 & 875.3 & 995   & 1136 \\
    \hline
    \textbf{November} & 452.825 & 797.9 & 864   & 825   & 1032 \\
    \hline
    \textbf{December} & 459.0528 & 825.4 & 932.2 & 812   & 1090 \\
    \hline
          &       &       &       &       &  \\
    \hline
    \textbf{Total Energy} & 7430.307 & 9486.4 & 9801.4 & 11132 & 12420 \\
    \hline
    \textbf{} &       &       &       &       &  \\
    \hline
    \textbf{MWh/MW} & 1486.061 & 1897.28 & 1960.28 & 2226.4 & 2484 \\
    \hline
          &       &       &       &       &  \\
    \hline
    \textbf{CUF} & 0.169642 & 0.216584 & 0.223776 & 0.254155 & 0.283562 \\
    \hline
    \end{tabular}%
    %% All above Code is copied from Excel to LATEX Add-In 
 
    \label{tabc2h5}%
\end{table}
\\
The Fig \ref{figc2h10} shows the monthly solar generations for different types of tracking mechanisms, and we see that the Dual Axis Tracker gives the best performance while Fixed Tilt gives the worst.
\\
\begin{figure}[H]
\centering
\includegraphics[scale=0.75]{Backbone1}
\caption{Monthly Solar Generation Comparison of different Tracking Mechanisms}
\label{figc2h10} %% to refer use, \ref{}
\end{figure}
\\
The Fig \ref{figc2h11} represents the CUF comparison of different tracking mechanisms. We observe that the performance of FT, ST, SA and DA improves in this same order.

\\

\begin{figure}[H]
\centering
\includegraphics[scale=0.65]{Backbone3}
\caption{CUF Comparison of different Tracking Mechanisms}
\label{figc2h11} %% to refer use, \ref{}
\end{figure}

\newpage

\section{Generation and Weather Data Resources}
\
\
\
\
The historical generation data can be sort from SLDC, but it has a resolution of one day. To get higher resolution generation data of the order of days or hours, we have to take it directly from the Solar PV Plant SCADA system. But in most of the cases the plant logs only daily data for saving memory of the SCADA systems. I have collected generation data from the BACKBONE plant at a daily resolution.\\

Similarly historical weather data is available at Meteorological Institutes Website, they will usually have data at almost all resolutions, but you will have to pay for the data. Also, some web linked softwares like Meteonorm alsso provide quality weather data at almost all resolutions at a fee. However, the data obtained from these sources would not be cent percent accurate for the exact plant location. Hence, weather data logged into the plant SCADA system is the most reliable weather data we can ever get. I have received only Global Horizoltal Radiation (GHI) from Backbone as there is no provision for temperature and wind speed logging.





\chapter{Basics of Solar Energy}
\input{Chapters/Ch3}

\chapter{Solar Photovoltaic Energy Estimation}
\input{Chapters/Ch4}

\chapter{Wind Turbine Energy Estimation}
\input{Chapters/Ch5}

\chapter{Forecasting Techniques in Brief}
\input{Chapters/Ch6}

\chapter{Generation Forecasting using ARIMA}
\input{Chapters/Ch7}

\chapter{Generation Forecasting using ANN}
\input{Chapters/Ch8}

\chapter{Generation Forecasting using WRF}
%% Chapter 9 : Generation Forecasting using WRF

\section{Introduction to NWP}
\
\
\
\
NWP stands for Numerical Weather Prediction. It uses mathematical models of the atmospheric and oceanic physical processes to predict the weather based on current weather conditions. These numerical solutions are possible due to the the advent of computer simulations. Therefore, though first NWP simulations were attempted in the 1920's realistic results were produced in the 1950's when computer simulation technology improved.\\

Many different global (predicting weather for entire earth) and regional (predicting weather for a particular region) forecast models are run in different countries worldwide. The current weather data for the initialization of these NWP models are acquired from radiosondes, weather satellites and other observing systems.\\

The NWP models use systems of differential equations based on the laws of physics, fluid motion, and chemistry, and use a coordinate system which divides the planet into a 3D grid. Winds, heat transfer, solar radiation, relative humidity, and surface hydrology are calculated within each grid cell, and the interactions with neighboring cells are used to calculate atmospheric properties in the future. These models can be used for both short-term forecasting for predicting weather as well as long-term forecasts for understanding the trend of climate change.\\

The mathematical models themselves are are chaotic due to the presence of partial differential equations that govern the atmospheric and oceanic physical processes. To solve these complex and chaotic equations with the vast datasets (initializing dynamic weather data and static geographical data)NWP's require sumpercomputer architectures for computing solutions. On the contrary, it is impossible to solve these equations exactly even with huge supercomputers, and the errors in the simulation grow with time. Moreover, parameterizations (replacing processes that are too small-scale or complex to be physically represented in the model by a simplified process) for various physical processes is also required for faster and more accurate simulations. Hence, with the present understanding of the NWP's, we can assume that accurate forecasts are generated to about 14 days even with perfectly accurate initializing data and a perfect model.\\

\section{The WRF NWP Model} 
\
\
\
\
The WRF or the Weather Research and Forecasting is a NWP system developed by a conglomerate of atmospheric and oceanic research institutions. The principal institutions are: National Center for Atmospheric Research (NCAR), the National Oceanic and Atmospheric Administration (represented by the National Centers for Environmental Prediction (NCEP) and the (then) Forecast Systems Laboratory (FSL)), the Air Force Weather Agency (AFWA), the Naval Research Laboratory (NRL), the University of Oklahoma (OU), and the Federal Aviation Administration (FAA). The bulk of the work on the model has been performed or supported by NCAR, NOAA, and AFWA.\\

The WRF has two dynamical (computational) cores (or solvers; ARW [Advanced Research WRF] developed by NCAR and the WRF-NMM [Nonhydostatic Mesoscale Model] developed by NCEP), a data assimilation system, and a software architecture allowing for parallel computation and system extensibility. The model serves a wide range of meteorological applications across scales ranging from meters to thousands of kilometers.

Being an open-source and community driven software, it has been adopted by many forecasting centers internationally. Moreover, there are about 23,000 registered WRF users in over 150 countries making the community strong and vibrant; which extends to the forums making them rich n content and hence easy to debug WRF simulation problems.\\


\section{WRF Software Components}
\
\
\
\
The Fig (\ref{figc10h1}) shows the software architecture of the WRF-ARW system.

\begin{figure}[H]
\centering
\includegraphics[scale=0.30]{WRF_1}
\caption{WRF-ARW Software Architecture}
\label{figc10h1} %% to refer use, \ref{}
\end{figure}

The brief overview of the major WRF system programs is as follows;\\

\begin{enumerate}

\item \blindtext 	\textbf{WPS:} This program is used primarily for real-data simulations. Its functions include 1) defining
simulation domains; 2) interpolating terrestrial data (such as terrain, landuse, and soil types) to the simulation domain; and 3) degribbing and interpolating meteorological data
from another model to this simulation domain.

\item \blindtext \textbf{WRF-DA:} This program is optional, but can be used to ingest observations into the interpolated
analyses created by WPS. It can also be used to update WRF model's initial conditions
when the WRF model is run in cycling mode.

\item \blindtext \textbf{ARW Solver:} This is the key component of the modeling system, which is composed of several
initialization programs for idealized, and real-data simulations, and the numerical
integration program.

\item \blindtext \textbf{Post-Processing & Visualization Tools:} Several programs are supported, including RIP4 (based on NCAR Graphics), NCAR
Graphics Command Language (NCL), and conversion programs for other readily
available graphics packages like GrADS. Program VAPOR, Visualization and Analysis Platform for Ocean, Atmosphere, and
Solar Researchers, is a 3-dimensional data visualization
tool, and it is developed and supported by the VAPOR team at NCAR.Program MET, Model Evaluation Tools , is
developed and supported by the Developmental Test bed Center at NCAR.

\end{enumerate}

We will not be using the WRF-DA and the Post-Processing & Visualization Tools  software as we will not be doing data assimilation, and e have developed a visualization and data extraction application in MATLAB to suit our needs. We will be using the WPS and the WRF-ARW Solver for performing day-ahead short-term weather forecasting for a real data case. \\

The detailed overview of the component programs of the WPS and the WR-ARW solver along with the step-wise implementation will be discussed in the subsequent sections of the text.\\

\newpage

\section{Real Data Case Short-Term Weather Forecasting using WPS and WRF-ARW Solver}
\
\
\
\
The Fig (\ref{figc10h2}) illustrates the step-wise implementation of the WPS and WRf programs for simulating a real data case.

\begin{figure}[H]
\centering
\includegraphics[scale=0.9]{WRF_22}
\caption{WPS and WRF Step-Wise Implementation Scheme}
\label{figc10h2} %% to refer use, \ref{}
\end{figure}

\subsection{WPS Program Components}
\
\
\
\
The WRF Preprocessing System (WPS) is a set of three programs whose collective role is
to prepare input to the real program for real-data simulations. Each of the programs
performs one stage of the preparation: geogrid defines model domains and interpolates
static geographical data to the grids; ungrib extracts meteorological fields from GRIBformatted
files; and metgrid horizontally interpolates the meteorological fields extracted
by ungrib to the model grids defined by geogrid. The work of vertically interpolating
meteorological fields to WRF eta levels is performed within the real program.\\

\begin{enumerate}

\item \blindtext \textbf{Program Geogrid:} The purpose of geogrid is to define the simulation domains, and interpolate various
terrestrial data sets to the model grids. The simulation domains are defined using
WPS
WRF-ARW V3: User’s Guide 3-3
information specified by the user in the “geogrid” namelist record of the WPS namelist
file, namelist.wps. 	

\item \blindtext \textbf{Program Ungrib:} The ungrib program reads GRIB files, "degribs" the data, and writes the data in a simple
format called the intermediate format. The GRIB files contain time-varying meteorological
fields and are typically from another regional or global model, such as NCEP's NAM or
GFS models. The ungrib program can read GRIB Edition 1 and, if compiled with a
"GRIB2" option, GRIB Edition 2 files. 

\item \blindtext \textbf{Program Metgrid:} The metgrid program horizontally interpolates the intermediate-format meteorological
data that are extracted by the ungrib program onto the simulation domains defined by the
geogrid program. The interpolated metgrid output can then be ingested by the WRF real
program.

\end{enumerate}

\subsection{WRF Program Components}
\
\
\
\
The WRF model is a fully compressible and nonhydrostatic model (with a run-time
hydrostatic option). Its vertical coordinate is a terrain-following hydrostatic pressure
coordinate. The grid staggering is the Arakawa C-grid. The model uses the Runge-Kutta
2nd and 3rd order time integration schemes, and 2nd to 6th order advection schemes in
both the horizontal and vertical. It uses a time-split small step for acoustic and gravitywave
modes. The dynamics conserves scalar variables.

\begin{enumerate}

\item \blindtext \textbf{real.exe:} It converts the output files from metgrid program of the WPS system into a format which can be used as an initialization for the wrf.exe	

\item \blindtext \textbf{wrf.exe:} It is the numerical integration program which solves the partial differential equations of the atmospheric and oceanic processes to compute the weather forecasts.	

\end{enumerate}

\newpage

\section{Computing Platform to Run WRF}
\
\
\
\
For running the the WRF software a 4-Node Raspberry-Pi2 (RPi) cluster has been developed, to provide for the distributed computation environment which makes the WRF simulations faster.
  
\subsection{Raspberry-Pi2 Micro-Computer}
\
\
\
\
It is a single-board micro-computer developed by the Raspberry Pi Foundation in the United Kingdom. Its primary purpose is to serve as a low priced tool for teaching and learning computer science. However, due its inexpensiveness and the broad spectrum of capabilities has made it a favourite of hobbyists, computer enthusiasts, students and researchers for embedded development. As it has an ARMv7 processor, it can run a full range of ARM GNU/Linux distributions.\\

\textbf{Hardware}:

\begin{itemize}

\item A 900MHz Quad-Core ARM Cortex-A7 CPU

\item 1GB RAM

\item 4 USB Ports

\item 40 GPIO (General Purpose Input/Output) pins

\item Full HDMI Port

\item Ethernet Port

\item Combined 3.5mm Audio Jack and Composite Video

\item Camera Interface (CSI)

\item Display Interface (DSI)

\item Micro SD Card slot

\item VideoCore IV Graphics Core	

\end{itemize}

\textbf{Software (Operating Systems):}:

\begin{itemize}

\item RASPBIAN

\item UBUNTU MATE

\item SNAPPY UBUNTU CORE

\item WINDOWS 10 IOT CORE

\item OSMC

\item OPENLECPINETRISC OS

\end{itemize}

\subsection{Setting up Raspberry Pi Cluster for WRF}
\
\
\
\
The Fig (\ref{figc10h3}) gives the step-wise instructions to develop a cluster of Raspberry-Pi2 micro-computers to run the WRF software

\begin{figure}[H]
\centering
\includegraphics[scale=1]{WRF_33}
\caption{Flow Diaram for Setting up Raspberry Pi Cluster for WRFe}
\label{figc10h3} %% to refer use, \ref{}
\end{figure}

\textbf{Set up a Single RPi:
}

\begin{enumerate} 

\item \blindtext Components required are: 1 × Raspberry Pi, 1 × 32GB Micro SD Card, 1 × Card Reader, 1 × USB Power Cable, 1 × USB Power Hub, 1 × HDMI Cable, 1 × USB Mouse, 1 × USB Keyboard and  1 × HDMI compatible Display.

\item \blindtext Download the RASPBIAN-OS from the Raspberry Pi website (free).

\item \blindtext Download SD Formatter Software from internet (free).

\item \blindtext Download Win32DiskImager Software from the internet (free).

\item \blindtext Format the 32GB Micro SD Card using SD Formatter software.

\item \blindtext Write the downloaded RASPBIAN-OS image to the 32GB Micro SD Card using the Win32DiskImager software.

\item \blindtext Insert the SD card in the RPi.

\item \blindtext Connect the USB Power Hub to the Power source, and the USB Power Cable to the USB Power Hub.

\item \blindtext Attach the USB mouse and keyboard to the USB ports of the RPi.

\item \blindtext Connect the HDMI cable from the RPi to the display screen.

\item \blindtext Power up the RPi by connecting the USB Power Cable to the Micro USB Power port of the Rpi.

\item \blindtext The Rpi will boot; Username – pi and Password – raspberry.

\item \blindtext Go to the configuration menu and do the following:

\begin{enumerate} 

	\item \blindtext Change the Hostname to Pi01

	\item \blindtext Enable SSH

	\item \blindtext Expand storage system
	
	\item \blindtext You may or may not overclock the RPi
	
\end{enumerate}	

\item \blindtext Reboot the Pi

\item \blindtext Shutdown the Rpi.


\end{enumerate}

\textbf{Remote Access RPi:
}\\

Remote access means, able to control RPi without additional display, mouse and keyboard. We will control the RPi via our workstation (Laptop/Desktop) using SSH (Secure Shell).

\begin{enumerate}

\item \blindtext 

\item \blindtext Components required are: LAN Cable
Download Putty software from the internet (free).

\item \blindtext Find out the IP address of the Ethernet port of the workstation and note it down.

\item \blindtext Take out the SD Card from the RPi, using the SD Card Reader to read into its boot partition and add xxx.xxx.xxx.RPi::xxx.xxx.xxx.PC at the end of the cmdline.txt file. Where the first part is the IP static address of the RPi which is set to be in the network space of the workstation’s IP address, and the second part after the double colons is the IP address of the workstation which was found out earlier.

\item \blindtext Now put the SD Card back into the RPi connect it to the workstation via a LAN cable and power it up.

\item \blindtext Open Putty software.

\item \blindtext In the Hostname field enter the Static IP address of our RPi, in the Port field enter 22, and give a name to this configuration and save with the hostname of our RPi i.e. Pi01.

\item \blindtext Now select the hostname of the RPi from the list, press Load and the press open.

\item \blindtext A terminal window will open, giving us a terminal access to our RPi.


\end{enumerate}

\textbf{Share Internet with RPi 
}

\begin{enumerate}

\item \blindtext From now on we will be using the RPi through SSH.

\item \blindtext We will share the internet (WiFi) of workstation with the RPi through the Ethernet port where it is connected.

\item \blindtext In a Windows system, this can be done by going to the \textit{Networks and Sharing} Option of the Control Panel.

\item \blindtext Go to the Change Adapter Settings page.

\item \blindtext There Right-Click on the WiFi icon which shows internet connectivity and hit Properties.

\item \blindtext Navigate to the \textit{Sharing} Tab and select the\textit{ Allow other users to connect through this computer’s Internet connection} checkbox.

\item \blindtext Now Right-Click on the \textit{LAN Adapter} and hit \textit{Properties}. Double click the \textit{IPV4} option and verify that some dynamic IP is populated. The IP address of the RPi should be within this IP addresses range.

\item \blindtext After this you can power up the RPi and connect to the workstation via the LAN cable.

\item \blindtext Now using Putty’s terminal we can have internet access in the RPi which is shared from the workstation. (Check connectivity using ping www.google.com)


\end{enumerate}

\textbf{Build WRF Software on RPi
}

\begin{enumerate}

\item \blindtext Download the TAR file of the WRF (Weather Research and Forecasting) and WPS (WRF Pre-processing System) from the WRF model site. Also download static geographic data used along with WPS and WRF from the UCAR EDU website.

\item \blindtext Download Filezilla Software from the internet (free).
 
\item \blindtext Run the RPi through SSH on the workstation, transfer the downloaded TAR files from the workstation to the RPi using the Filezilla software.

\item \blindtext Use the WRF Build Scripts developed to build and install WRF system on the RPi.

\item \blindtext Power off the RPi and remove the SD Card.

\item \blindtext Using the SD Card Reader and the Win32DiskImager; create an image of the SD Card, and store it in the workstation

\end{enumerate}

\textbf{Set up RPi Cluster
}

\begin{enumerate}



\item \blindtext Components required are: 4 × Raspberry Pi, 4 × 32GB Micro SD Cards, 1 × Card Reader, 4 × USB Power Cables, 4 × LAN Cables, 2 × USB Power Hub, 1 × Extension Box.

\item \blindtext Using the SD Card reader and the Win32DiskImager software, burn the image which was earlier stored on the workstation (with installed WRF) onto to the rest of the SD Cards.

\item \blindtext Now, for each of the new SD Cards using the SD Card Reader read into its boot partition and add xxx.xxx.xxx.RPi::xxx.xxx.xxx.PC at the end of the cmdline.txt file. Where the first part is the IP static address of the RPi which is set to be in the network space of the workstation’s IP address, and the second part after the double colons is the IP address of the workstation which was found out earlier. Make sure that all the IP addresses are unique.

\item \blindtext Put the SD Cards in the respective RPi’s.

\item \blindtext Connect all the RPi’s and the workstation to the Network Switch via LAN Cables.

\item \blindtext Connect all the RPi’s to the Power USB Hubs using the USB Power Cables.

\item \blindtext Connect the workstation, the Network Switch and the Power USB Hubs to the Power Source, switch on the Power Source.

\item \blindtext Using Putty softare to SSH into all the RPi’s create a passwordless SSH between them.

\item \blindtext Create a machine file in each of the RPi’s home directory.

\item \blindtext The RPi cluster is ready to use.

\end{enumerate}

The Fig (\ref{figc10h4}) illustrates the schematic of the 4-Node Raspberry-Pi2 Cluster developed for running the WRF software in a cluster environment so as to use its parallel computation feature  to its full potential.

\begin{figure}[H]
\centering
\includegraphics[scale=1]{RPiCluster_IMG1}
\caption{Schematic: 4-Node Raspberry-Pi2 Cluster }
\label{figc10h4} %% to refer use, \ref{}
\end{figure}

The results of energy forecasting using WRF software are presented in the next section.In addition, a GUI based application for visualization and extraction of forecasted weather variables from the WRF NETCDF output files is developed in MATLAB, the application GUI can be found in the Appendix. 

\newpage

\section{Results}

\subsection{4-Node Raspberry-Pi2 Cluster for Running WRF}
\
\
\
\
The Fig (\ref{}) shows the working 4-Node Raspberry-Pi2 Cluster on my desk in GERMI.

\begin{figure}[H]
\centering
\includegraphics[scale=0.20]{WRFCluster1}
\caption{Schematic: 4-Node Raspberry-Pi2 Cluster }
\label{figc10h4} %% to refer use, \ref{}
\end{figure}



\subsection{WRF Namelist Files}
\
\
\
\
\
Name-List files provide for options used to customize the WRF simulation, it is through changing the parameters in these files that the WRF can be run as desired by the user. The WRF software has two name-list files: namelist.wps and namelist.output which are given in the Tables (\ref{WRFTab1},\ref{WRFTab2}).\\

The namelist.wps file is used by the WPS software. It providing options for customizing the geogrid.exe, ungrib.exe and metgrid.exe application components.\\

The namelist.output file is used by the WRF software. It provides options for customizing the model run duration, numerical integration parameters, physics options and nesting parameters.


\newpage

\begin{table}[H]
  \centering
  \caption{namelist.wps Options Table}
    \begin{tabular}{|l|l|}
    \hline
    \multicolumn{2}{|c|}{\textbf{namelist.wps-GSEC 1MW}} \bigstrut\\
    \hline
    \multicolumn{1}{|c|}{\textbf{Namelist Variables}} & \multicolumn{1}{c|}{\textbf{Values}} \bigstrut\\
    \hline
    \textbf{\&share} &  \bigstrut\\
    \hline
    wrf\_core & ARW', \bigstrut\\
    \hline
    max\_dom  & 2, \bigstrut\\
    \hline
    interval\_seconds  & 21600 \bigstrut\\
    \hline
    \textbf{\&geogrid} &  \bigstrut\\
    \hline
    parent\_id & 1,   1, \bigstrut\\
    \hline
    parent\_grid\_ratio & 1,   3, \bigstrut\\
    \hline
    i\_parent\_start & 1,   2, \bigstrut\\
    \hline
    j\_parent\_start & 1,   2, \bigstrut\\
    \hline
    s\_we               & 1,   1, \bigstrut\\
    \hline
    e\_we               & 13,  28, \bigstrut\\
    \hline
    s\_sn               & 1,   1, \bigstrut\\
    \hline
    e\_sn               & 13,  28, \bigstrut\\
    \hline
    geog\_data_res      & 10m','10m', \bigstrut\\
    \hline
    dx & 3000, \bigstrut\\
    \hline
    dy  & 3000, \bigstrut\\
    \hline
    map\_proj  & 'lambert', \bigstrut\\
    \hline
    ref\_lat  &  23.275, \bigstrut\\
    \hline
    ref\_lon  & 72.682, \bigstrut\\
    \hline
    truelat1 &  30.0, \bigstrut\\
    \hline
    truelat2 & 60.0, \bigstrut\\
    \hline
    stand\_lon & -98.0, \bigstrut\\
    \hline
    geog\_data\_path  & '/mnt/USB/WPS_GEOG/' \bigstrut\\
    \hline
    \textbf{\&ungrib} &  \bigstrut\\
    \hline
    out\_format  & 'WPS', \bigstrut\\
    \hline
    prefix  & 'FILE', \bigstrut\\
    \hline
    \textbf{\&metgrid} &  \bigstrut\\
    \hline
    fg\_name & 'FILE' \bigstrut\\
    \hline
    io\_form\_metgrid & 2, \bigstrut\\
    \hline
    \end{tabular}%
  \label{WRFTab1}%
\end{table}%

\newpage

\begin{table}[H]
  \centering
  \caption{namelist.output Options Table}
    \begin{tabular}{|l|l|}
    \hline
    \multicolumn{2}{|c|}{\textbf{namelist.output-GSEC 1MW}} \bigstrut\\
    \hline
    \multicolumn{1}{|c|}{\textbf{Namelist Variables}} & \multicolumn{1}{c|}{\textbf{Values}} \bigstrut\\
    \hline
    \textbf{\&time\_control} &  \bigstrut\\
    \hline
    run\_days & 0, \bigstrut\\
    \hline
    run\_hours                            & 24, \bigstrut\\
    \hline
    run\_minutes                          & 0, \bigstrut\\
    \hline
    run\_seconds                          & 0, \bigstrut\\
    \hline
    end\_hour                             & 00,   00,   12, \bigstrut\\
    \hline
    end\_minute                           & 00,   00,   00, \bigstrut\\
    \hline
    end\_second                           & 00,   00,   00, \bigstrut\\
    \hline
    interval\_seconds                     & 21600 \bigstrut\\
    \hline
    input\_from\_file                      & .true.,.true.,.true., \bigstrut\\
    \hline
    history\_interval                     & 15,  15,   60 \bigstrut\\
    \hline
    frames\_per\_outfile                   & 1000, 1000, 1000, \bigstrut\\
    \hline
    restart  & .false., \bigstrut\\
    \hline
    restart\_interval & 5000, \bigstrut\\
    \hline
    io\_form\_history                      & 2 \bigstrut\\
    \hline
    io\_form\_restart                      & 2 \bigstrut\\
    \hline
    io\_form\_input                        & 2 \bigstrut\\
    \hline
    io\_form\_boundary                     & 2 \bigstrut\\
    \hline
    debug\_level                          & 0 \bigstrut\\
    \hline
    \textbf{\&domains} &  \bigstrut\\
    \hline
    time\_step        & 18,       \bigstrut\\
    \hline
    time\_step\_fract\_num                  & 0, \bigstrut\\
    \hline
    time\_step\_fract\_den                  & 1, \bigstrut\\
    \hline
    max\_dom                              & 2, \bigstrut\\
    \hline
    e\_we                                 & 13,    28,   94, \bigstrut\\
    \hline
    e\_sn                                 & 13,    28,    91, \bigstrut\\
    \hline
    e\_vert                               & 28,    28,    28, \bigstrut\\
    \hline
    p\_top\_requested                      & 5000, \bigstrut\\
    \hline
    num\_metgrid\_levels                   & 32, \bigstrut\\
    \hline
    \end{tabular}%
  \label{WRFTab2}%
\end{table}%

\newpage

\begin{table}[H]
  \centering
  
    \begin{tabular}{|l|l|}
    \hline
    \multicolumn{2}{|c|}{\textbf{namelist.output-GSEC 1MW}} \bigstrut\\
    \hline
    \multicolumn{1}{|c|}{\textbf{Namelist Variables}} & \multicolumn{1}{c|}{\textbf{Values}} \bigstrut\\
    \hline
    num\_metgrid\_soil\_levels              & 4, \bigstrut\\
    \hline
    dx  & 3000, 1000,  3333.33, \bigstrut\\
    \hline
    dy  & 3000, 1000,  3333.33, \bigstrut\\
    \hline
    grid\_id                              & 1,     2,     3, \bigstrut\\
    \hline
    parent\_id                            & 0,     1,     2, \bigstrut\\
    \hline
    i\_parent\_start                       & 1,     2,    30, \bigstrut\\
    \hline
    j\_parent\_start                       & 1,     2,    30, \bigstrut\\
    \hline
    parent\_grid\_ratio                    & 1,     3,     3, \bigstrut\\
    \hline
    parent\_time\_step\_ratio               & 1,     3,     3, \bigstrut\\
    \hline
    feedback  & 1, \bigstrut\\
    \hline
    smooth\_option                        & 0 \bigstrut\\
    \hline
    \textbf{\&physics} &  \bigstrut\\
    \hline
    mp\_physics                           & 3,     3,     3, \bigstrut\\
    \hline
    ra\_lw\_physics                        & 1,     1,     1, \bigstrut\\
    \hline
    ra\_sw\_physics                        & 1,     1,     1, \bigstrut\\
    \hline
    radt                                 & 30,    30,    30, \bigstrut\\
    \hline
    sf\_sfclay\_physics                    & 1,     1,     1, \bigstrut\\
    \hline
    sf\_surface\_physics                   & 2,     2,     2, \bigstrut\\
    \hline
    bl\_pbl\_physics                       & 1,     1,     1, \bigstrut\\
    \hline
    bldt & 0,     0,     0, \bigstrut\\
    \hline
    cu\_physics                           & 1,     1,     0, \bigstrut\\
    \hline
    cudt                                 & 5,     5,     5, \bigstrut\\
    \hline
    isfflx                               & 1, \bigstrut\\
    \hline
    ifsnow                               & 0, \bigstrut\\
    \hline
    icloud                               & 1, \bigstrut\\
    \hline
    surface\_input\_source                 & 1, \bigstrut\\
    \hline
    num\_soil\_layers                      & 4, \bigstrut\\
    \hline
    sf\_urban\_physics                     & 0,     0,     0, \bigstrut\\
    \hline
    maxiens                              & 1, \bigstrut\\
    \hline
    \end{tabular}%
  \label{tab:addlabel}%
\end{table}%

\newpage

\begin{table}[H]
  \centering
  
    \begin{tabular}{|l|l|}
    \hline
    \multicolumn{2}{|c|}{\textbf{namelist.output-GSEC 1MW}} \bigstrut\\
    \hline
    \multicolumn{1}{|c|}{\textbf{Namelist Variables}} & \multicolumn{1}{c|}{\textbf{Values}} \bigstrut\\
    \hline
    maxens & 3, \bigstrut\\
    \hline
    maxens2 & 3, \bigstrut\\
    \hline
    maxens3                              & 16, \bigstrut\\
    \hline
    ensdim                               & 144, \bigstrut\\
    \hline
    \textbf{\&fdda} &  \bigstrut\\
    \hline
    \textbf{\&dynamics} &  \bigstrut\\
    \hline
    w\_damping                            & 0, \bigstrut\\
    \hline
    diff\_opt                             & 1, \bigstrut\\
    \hline
    km\_opt                               & 4, \bigstrut\\
    \hline
    km\_opt                               & 0,      0,      0, \bigstrut\\
    \hline
    diff\_6th\_factor & 0.12,   0.12,   0.12, \bigstrut\\
    \hline
    base\_temp                            & 290 \bigstrut\\
    \hline
    damp\_opt                             & 0, \bigstrut\\
    \hline
    zdamp & 5000.,  5000.,  5000., \bigstrut\\
    \hline
    dampcoef & 0.2,    0.2,    0.2 \bigstrut\\
    \hline
    khdif & 0,      0,      0, \bigstrut\\
    \hline
    kvdif & 0,      0,      0, \bigstrut\\
    \hline
    non\_hydrostatic                      & .true., .true., .true., \bigstrut\\
    \hline
    moist\_adv\_opt                        & 1,      1,      1,      \bigstrut\\
    \hline
    scalar\_adv\_opt                       & 1,      1,      1,      \bigstrut\\
    \hline
    \textbf{\&bdy\_control} &  \bigstrut\\
    \hline
    spec\_bdy\_width                       & 5, \bigstrut\\
    \hline
    spec\_zone                            & 1, \bigstrut\\
    \hline
    relax\_zone                           & 4, \bigstrut\\
    \hline
    specified & .true., .false.,.false., \bigstrut\\
    \hline
    nested & .false., .true., .true., \bigstrut\\
    \hline
    \textbf{\&namelist\_quilt} &  \bigstrut\\
    \hline
    nio\_tasks\_per\_group  & 0, \bigstrut\\
    \hline
    nio\_groups  & 1, \bigstrut\\
    \hline
    \end{tabular}%
  \label{tab:addlabel}%
\end{table}%


\subsection{WRF-NETCDF Visualization and Extraction App}
\
\
\
\
The Fig (\ref{WRFResImg1},\ref{WRFResImg2},\ref{WRFResImg3},\ref{WRFResImg4},\ref{WRFResImg5}) illustrate the Latitude-Longitude Grid (developed around the GSEC 1MW SPVP), the 10m wind speeds in U direction, the 10m wind speeds in V direction, the 2m temperature, and the short-wave downward flux superimposed on the latitude-longitude grid respectively on the $1^{st}$ of June, 2016 at 06:15:00 UTC.

\begin{figure}[H]
\centering
\includegraphics[scale=0.5]{LAT_LONG_GSEC}
\caption{GSEC 1MW WRF Simulation-Latitude Longitude Grid}
\label{WRFResImg1} %% to refer use, \ref{}
\end{figure}

\begin{figure}[H]
\centering
\includegraphics[scale=0.5]{U10_NETCDF_GSEC}
\caption{GSEC 1MW WRF Simulation-10m Wind Velocity (u-Component) Grid}
\label{WRFResImg2} %% to refer use, \ref{}
\end{figure}

\begin{figure}[H]
\centering
\includegraphics[scale=0.5]{V10_NETCDF_GSEC}
\caption{ GSEC 1MW WRF Simulation-10m Wind Velocity (v-Component) Grid}
\label{WRFResImg3} %% to refer use, \ref{}
\end{figure}

\begin{figure}[H]
\centering
\includegraphics[scale=0.5]{T2_NETCDF_GSEC}
\caption{ GSEC 1MW WRF Simulation-2m Temperature Grid}
\label{WRFResImg4} %% to refer use, \ref{}
\end{figure}

\begin{figure}[H]
\centering
\includegraphics[scale=0.5]{SWDOWN_NETCDF_GSEC}
\caption{ GSEC 1MW WRF Simulation- Short Wave Downward Flux Grid}
\label{WRFResImg5} %% to refer use, \ref{}
\end{figure}


\subsection{Forecasting Results}
\
\
\
\
The WRF software has been run for the entire month of June,2016 with initialization data acquired from NCEP FTP server for the region consisting of the GSEC 1MW SPVP, Gandhinagar, Gujarat at a spatial resolution of 1km and a temporal resolution of 15 minutes. The results thus obtained have been extracted from the NETCDF wrf.out files into excel files for the latitude and longitude of the GSEC SPVP using the WRF-NETCDF Visualization and Extraction App. The variables extracted are the 10m Wind Speed (U and V direction), 2m Temperature and the Short-Wave Downward Flux. Out of these the wind speed files have been processed to give  a single wind speed for the desired grid cell, and the temperature in Kelvin has been converted to $^{\circ}$C. These converted excel files have been fed as input to the solar energy estimation app and the energy forecast have been generated. The following sections illustrate graphs of the weather variables generated by WRF and the energy forecast generated from the solar energy estimation app, and their comparison with the actual weather and energy output data of the GSEC SPVP for the month of June, 2015 (due to unavailability of recent data). The graphs for the dates 5, 10, 15, 20, 25 have been shown.

\newpage


\subsubsection{5^{th} \textbf{June, 2016:}}
\
\
\
\

\begin{figure}[H]
\centering
\includegraphics[scale=0.5]{WRFw5}
\caption{Comparision Of Actual And WRF Generated Wind Speed For 5th June}
\label{WRFResImg6} %% to refer use, \ref{}
\end{figure}

\begin{figure}[H]
\centering
\includegraphics[scale=0.5]{WRFt5}
\caption{Comparision Of Actual And WRF Generated Temperature For 5th June}
\label{WRFResImg7} %% to refer use, \ref{}
\end{figure}

\begin{figure}[H]
\centering
\includegraphics[scale=0.5]{WRFi5}
\caption{Comparision Of Actual And WRF Generated Irradiance For 5th June}
\label{WRFResImg8} %% to refer use, \ref{}
\end{figure}

\begin{figure}[H]
\centering
\includegraphics[scale=0.5]{WRFe5}
\caption{Comparision Of Actual And WRF Generated Energy For 5th June}
\label{WRFResImg9} %% to refer use, \ref{}
\end{figure}


\subsubsection{10^{th} \textbf{June, 2016:}}
\
\
\
\

\begin{figure}[H]
\centering
\includegraphics[scale=0.5]{WRFw10}
\caption{ Comparision Of Actual And WRF Generated Wind Speed For 10th June }
\label{WRFResImg10} %% to refer use, \ref{}
\end{figure}

\begin{figure}[H]
\centering
\includegraphics[scale=0.5]{WRFt10}
\caption{Comparision Of Actual And WRF Generated Temperature For 10th June}
\label{WRFResImg11} %% to refer use, \ref{}
\end{figure}

\begin{figure}[H]
\centering
\includegraphics[scale=0.5]{WRFi10}
\caption{Comparision Of Actual And WRF Generated Irradiance For 10th June}
\label{WRFResImg12} %% to refer use, \ref{}
\end{figure}

\begin{figure}[H]
\centering
\includegraphics[scale=0.5]{WRFe10}
\caption{Comparision Of Actual And WRF Generated Energy For 10th June}
\label{WRFResImg13} %% to refer use, \ref{}
\end{figure}



\subsubsection{15^{th} \textbf{June, 2016:}}
\
\
\
\

\begin{figure}[H]
\centering
\includegraphics[scale=0.5]{WRFw15}
\caption{Comparision Of Actual And WRF Generated Wind Speed For 15th June }
\label{WRFResImg14} %% to refer use, \ref{}
\end{figure}

\begin{figure}[H]
\centering
\includegraphics[scale=0.5]{WRFt15}
\caption{Comparision Of Actual And WRF Generated Temperature For 15th June }
\label{WRFResImg15} %% to refer use, \ref{}
\end{figure}

\begin{figure}[H]
\centering
\includegraphics[scale=0.5]{WRFi15}
\caption{Comparision Of Actual And WRF Generated Irradiance For 15th June }

\label{WRFResImg16} %% to refer use, \ref{}
\end{figure}
\begin{figure}[H]
\centering
\includegraphics[scale=0.5]{WRFe15}
\caption{Comparision Of Actual And WRF Generated Energy For 15th June}
\label{WRFResImg17} %% to refer use, \ref{}
\end{figure}

\newpage

\subsubsection{20^{th} \textbf{June, 2016:}}
\
\
\
\

\begin{figure}[H]
\centering
\includegraphics[scale=0.5]{WRFw20}
\caption{Comparision Of Actual And WRF Generated Wind Speed For 20th June}
\label{WRFResImg18} %% to refer use, \ref{}
\end{figure}

\begin{figure}[H]
\centering
\includegraphics[scale=0.5]{WRFt20}
\caption{Comparision Of Actual And WRF Generated Temperature For 20th June}
\label{WRFResImg19} %% to refer use, \ref{}
\end{figure}

\begin{figure}[H]
\centering
\includegraphics[scale=0.5]{WRFi20}
\caption{Comparision Of Actual And WRF Generated Irradiance For 20th June}
\label{WRFResImg20} %% to refer use, \ref{}
\end{figure}

\begin{figure}[H]
\centering
\includegraphics[scale=0.5]{WRFe20}
\caption{Comparision Of Actual And WRF Generated Energy For 20th June}
\label{WRFResImg21} %% to refer use, \ref{}
\end{figure}


\subsubsection{25^{th} \textbf{June, 2016:}}
\
\
\
\

\begin{figure}[H]
\centering
\includegraphics[scale=0.5]{WRFw25}
\caption{Comparision Of Actual And WRF Generated Wind Speed For 25th June}
\label{WRFResImg22} %% to refer use, \ref{}
\end{figure}

\begin{figure}[H]
\centering
\includegraphics[scale=0.5]{WRFt25}
\caption{Comparision Of Actual And WRF Generated Temperature For 25th June}
\label{WRFResImg23} %% to refer use, \ref{}
\end{figure}

\begin{figure}[H]
\centering
\includegraphics[scale=0.5]{WRFi25}
\caption{Comparision Of Actual And WRF Generated Irradiance For 25th June}
\label{WRFResImg24} %% to refer use, \ref{}
\end{figure}

\begin{figure}[H]
\centering
\includegraphics[scale=0.5]{WRFe25}
\caption{Comparision Of Actual And WRF Generated Energy For 25th June}
\label{WRFResImg25} %% to refer use, \ref{}
\end{figure}

\subsubsection{Conclusion from Graphs}
\
\
\
\
The graphs shown in the previous sections show that the WRF is able to compute the weather variables approximately for the desired SPVP location. It is not able to model the sharp dips in the irradiances which are caused by cloud movement (prediction of cloud movements is poor in WRF). However, when the actual irradiance graphs have no significant sharp dips i.e. clouds are not present, then the WRF irradiance is slightly over estimated (due to poor modeling of aerosols). Overall, WRF shows potential for a day-ahead forecast which can be improved further by improved parameterization and appropriate physics option selection based on the region of forecast.

\chapter{Conclusion and Future Work}
\input{Chapters/Ch10}

\chapter{References}
%% Chapter 11: References

a
\newpage 
b
\newpage
c
\newpage
d
\newpage
e
\newpage
f
\newpage
g


%% References

%\bibliographystyle{plain}
%\bibliography{References}
%\nocite{*}

%% Appendix

\appendix
\chapter{Application Start GUIs }
\input{Chapters/Appendix8}

\chapter{Solar Energy Estimation App GUIs}
\input{Chapters/Appendix1}

\chapter{Wind Energy Estimation App GUIs }
%% Appendix2

\begin{figure}[H]
\centering
\includegraphics[scale=0.5]{WindCpCurveApp}
\caption{Cp Curve Generator App GUI}
\label{figApp2_1} %% to refer use, \ref{}
\end{figure}

\begin{figure}[H]
\centering
\includegraphics[scale=0.5]{WindDataAcqApp1}
\caption{Wind Data Acquisition App GUI}
\label{figApp2_2} %% to refer use, \ref{}
\end{figure}

\begin{figure}[H]
\centering
\includegraphics[scale=0.5]{WindPowerCurve}
\caption{Wind Turbine Power Curve Viewer GUI}
\label{figApp2_3} %% to refer use, \ref{}
\end{figure}

\begin{figure}[H]
\centering
\includegraphics[scale=0.5]{WindEnergyEstimationApp}
\caption{Wind Energy Estimation App GUI}
\label{figApp2_4} %% to refer use, \ref{}
\end{figure}

\begin{figure}[H]
\centering
\includegraphics[scale=0.5]{WindWeibull}
\caption{Weibull Distribution App GUI}
\label{figApp2_5} %% to refer use, \ref{}
\end{figure}

\begin{figure}[H]
\centering
\includegraphics[scale=0.5]{WindLossParameters}
\caption{Loss Parameters GUI}
\label{figApp2_6} %% to refer use, \ref{}
\end{figure}

\begin{figure}[H]
\centering
\includegraphics[scale=0.5]{WindWakeModels}
\caption{Wind Wake Models Data GUI}
\label{figApp2_7} %% to refer use, \ref{}
\end{figure}

\begin{figure}[H]
\centering
\includegraphics[scale=0.5]{WindResult}
\caption{Wind Energy Simulation Report GUI}
\label{figApp2_8} %% to refer use, \ref{}
\end{figure}

\chapter{ARIMA Forecasting App GUIs }
%% Appendix3

\begin{figure}[H]
\centering
\includegraphics[scale=1]{arima2}
\caption{ARIMA-Data Acquisition GUI}
\label{figApp3_1} %% to refer use, \ref{}
\end{figure}

\begin{figure}[H]
\centering
\includegraphics[scale=0.5]{arima3}
\caption{ARIMA-Model Identification GUI}
\label{figApp3_2} %% to refer use, \ref{}
\end{figure}

\begin{figure}[H]
\centering
\includegraphics[scale=0.65]{arima4}
\caption{ARIMA-Model Creation GUI}
\label{figApp3_3} %% to refer use, \ref{}
\end{figure}

\begin{figure}[H]
\centering
\includegraphics[scale=0.75]{arima5}
\caption{ARIMA-Model Simulation GUI}
\label{figApp3_4} %% to refer use, \ref{}
\end{figure}

\begin{figure}[H]
\centering
\includegraphics[scale=0.5]{arima6}
\caption{ARIMA-Model Estimation GUI}
\label{figApp3_5} %% to refer use, \ref{}
\end{figure}

\begin{figure}[H]
\centering
\includegraphics[scale=1]{arima8}
\caption{ARIMA-Model Estimate Viewer GUI}
\label{figApp3_5} %% to refer use, \ref{}
\end{figure}

\begin{figure}[H]
\centering
\includegraphics[scale=0.65]{arima7}
\caption{ARIMA=Model Forecasting GUI}
\label{figApp3_6} %% to refer use, \ref{}
\end{figure}

\chapter{ANN Forecasting App GUIs }
\input{Chapters/Appendix4}

\chapter{WRF NETCDF Visualization and Extraction App GUI }
%% Appendix5

%% Appendix6

\begin{figure}[H]
\centering
\includegraphics[scale=0.75]{WrfNetcdfApp}
\caption{WRF NETCDF Visualization and Extraction App GUI}
\label{figApp6_1} %% to refer use, \ref{}
\end{figure}

\chapter{Data Pre-Processing System App GUIs }
%% Appendix7

\begin{figure}[H]
\centering
\includegraphics[scale=.8]{DataPreProcSys}
\caption{Data Pre-Processing App GUI}
\label{figApp5_1} %% to refer use, \ref{}
\end{figure}


\end{document}
\end{document}
\end{document}
\end{document}