%% Equations for Thesis Chapter-Wise
\documentclass[12pt]{article}
\usepackage{amsmath}
\usepackage{amsfonts}
\usepackage{amssymb}
\usepackage[a4paper, width=150mm, top=25mm, bottom=25mm]{geometry}

\begin{document}



\textbf{\huge{Thesis Equations:}}\\

\textbf{\textit{\large{Chapter 3 Equations:}}}\\
%% Chapter 3 Equations Start







\begin{eqnarray*}



r_{p}=R_{p}\alpha_{p}

\\

\text{Where;}

\\


$r_{p}=[r(1)  r(2) ... r(p)]$
\\

R_{p} &=\begin{bmatrix}
    r(0) & r(1)  & \dots  & r(p-1) \\
    r(1) & r(0)  & \dots  & r(p-2) \\
    \vdots & \vdots & \ddots & \vdots \\
    r(p-1) & r(p-2) } & \dots  & r(0)
\end{bmatrix}


\\
$\alpha_{p}=[\alpha(1)  \alpha(2) ... \alpha(p)]$
\\
$r(1)=\frac{\widehat{\gamma}(1)}{\widehat{\gamma}(0)}$ & \text{; Auto-Correlation at Lag 1}
\\
\widehat{\gamma}(h)=\frac{1}{n}\sum_{t=1}^{n-h}\left(  X_{t}-\bar{X}  \right)\left( X_{t+h}- \bar{X}  \right) & \text{; Auto-Covariance at Lag h}

\\

\text{The AR parameters are obtained from the following equation;}

\\

\alpha_{p}}=R_{p}^{-1}r_{p}


\end{eqnarray*}



\begin{equation}
\label{planck}
E_{\lambda}=\frac{3.74\times10^{8}}{\lambda^{5} \left[\text{exp}\left( \frac{14,400}{\lambda T}\right)-1 \right]}
\end{equation}

\begin{equation}
\label{stefbolt}
E = A \sigma T^{4}
\end{equation}

\begin{equation}
\label{wien}
\lambda_{{max}}(\mu m)=\frac{2898}{T(K)}
\end{equation}

\begin{equation}
\label{diode}
I_{d} = I_{0}( e^{qV_{d}/kT}-1)
\end{equation}

\begin{equation}
\label{pv1}
I = I_{SC}-I_{d}
\end{equation}

\begin{equation}
\label{pv2}
I = I_{SC}-I_{0} \left(e^{qV/kT}-1 \right )
\end{equation}

\begin{equation}
\label{pv3}
V_{OC} = \frac{kT}{q}\ln \left (\frac{I_{SC}}{I_{0}}+1\right )
\end{equation}


\begin{equation}
\label{pv4}
I = I_{SC}-I_{0} \left(e^{38.9V}-1 \right )
\end{equation}

\begin{equation}
\label{pv5}
V_{OC} = 0.0257\ln \left (\frac{I_{SC}}{I_{0}}+1\right )
\end{equation}


\begin{equation}
\label{pv6}
I = (I_{SC}-I_{d})-\frac{V}{R_{P}}
\end{equation}

\begin{equation}
\label{pv7}
V_{d}=V+I.R_{S}
\end{equation}

\begin{equation}
\label{pv8}
I = I_{SC}-I_{0} \left\{\text{exp}\left[\frac{q(V+I.R_{S})}{kT} \right]-1 \right\}
\end{equation}

\begin{equation}
\label{pv9}
I = I_{SC}-I_{0} \left\{\text{exp}\left[\frac{q(V+I.R_{S})}{kT} \right]-1 \right\}-\left(\frac{V+I.R_{S}}{R_{P}}\right)
\end{equation}

\begin{equation}
\label{pv10}
I_{SC}=I+I_{d}+I_{P}
\end{equation}

\begin{equation}
\label{pv11}
V=V_{d}-IR_{S}
\end{equation}

\begin{equation}
\label{mod1}
V_{{module}}=n(V_{d}-IR_{S})
\end{equation}

\begin{equation}
\label{mod2}
I_{{module}}=p(I_{SC}-I_{d}-I_{P})
\end{equation}

\begin{equation}
\label{tefpv}
T_{{cell}}=T_{{amb}}+\left(\frac{NOCT-20^{\circ}}{0.8}\right)
.\ S
\end{equation}

\begin{equation}
\label{shade1}
V_{SH}=V_{n-1}-I(R_{P}+R_{S})
\end{equation}

\begin{equation}
\label{shade2}
V_{n-1}=\left(\frac{n-1}{n}\right)V
\end{equation}

\begin{equation}
\label{shade3}
V_{SH}=\left(\frac{n-1}{n}\right)V-I(R_{P}+R_{S})
\end{equation}

\begin{equation}
\label{shade4}
\triangledown V=\frac{V}{n}-I(R_{P}+R_{S})
\end{equation}

\\
%% Chapter 3 Equations End


\textbf{\textit{\large{Chapter 4 Equations:}}}\\
%% Chapter 4 Equations Start

\begin{equation}
\label{amr}
\text{Air Mass Ratio}\quad m=\frac{h_{2}}{h_{1}}=\frac{1}{\sin{\beta}}
\end{equation}

\begin{equation}
\label{eo}
d=1.5\times10^{8}\left \{\ 1+0.017 \sin \left[ \frac{360(n-93)}{365} \right] \right\} \ \text{km}
\end{equation}

\begin{equation}
\label{sdec}
\delta =23.45 \sin \left[ \frac{360}{365} (n-81)\right]  
\end{equation}

\begin{equation}
\label{altn}
\beta_{N}=90^{\circ}-L+\delta
\end{equation}

\begin{equation}
\label{tilt1}
\text{Tilt}=90^{\circ}-\beta_{N}
\end{equation}

\begin{equation}
\label{alt}
\sin{(\beta)}=\cos{(L)}\cos{(\delta)}cos{(H)}+\sin{(L)}\sin{(\delta)}
\end{equation}

\begin{equation}
\label{azm}
\sin{(\phi_{S})}=\frac{\cos{(\delta)}\sin{(H)}}{\cos{(\beta)}}
\end{equation}

\begin{equation}
\label{ha}
\text{Hour Angle}\ H=\left(\frac{15^{\circ}}{\text{hour}}\right).(\text{hours before solar noon})
\end{equation}

\begin{equation}
\label{hac}
\text{if,} \quad \cos{(H)}\geq\frac{\tan{(\delta)}}{\tan{(L)}}; \hspace{1cm} \text{then,} \ |\phi_{S}|\leq90^{\circ}; \hspace{1cm} \text{otherwise,} \ |\phi_{S}|>90^{\circ}
\end{equation}

\begin{equation}
\label{eot}
E= 9.87\sin{(2B)} -7.53\cos{(B)} -1.5\sin{(B)} \quad \text{minutes}
\end{equation}

\begin{equation}
\label{eotb}
B=\frac{360}{364}(n-81)
\end{equation}

\begin{equation}
\label{solartime}
\text{Solar Time (ST)}= \text{Clock Time (CT)}+ \frac{4 \ \text{min}}{\text{degree}}(\text{Local Time Meridian}- \text{Local Logitude})^{\circ} + E(\text{min})
\end{equation}

\begin{equation}
\label{sunrise1}
\sin{(\beta)}=\cos{(L)}\cos{(\delta)}cos{(H)}+\sin{(L)}\sin{(\delta)}=0
\end{equation}

\begin{equation}
\label{sunrise2}
\cos{(H)}=-\frac{\sin{(L)}\sin{(\delta)}}{\cos{(L)}\cos{(\delta)}}=-\tan{(L)}\tan{(\delta)}
\end{equation}

\begin{equation}
\label{sunrise3}
H_{SR}=\cos^{-1}{(-\tan{(L)}\tan{(\delta)})} \quad \text{(+ for sunrise)}
\end{equation}

\begin{equation}
\label{sunrise4}
\text{Sunrise(geometric)}= 12:00-\frac{H_{SR}}{15^{\circ}/h}
\end{equation}

\begin{equation}
\label{sunrise5}
Q=\frac{3.467}{\cos{(L)}\cos{(\delta)}\sin{(H_{SR})}} \quad (\text{min})
\end{equation}

\begin{equation}
\label{etr1}
I_{0}=\text{SC}. \left[1+0.0334\cos\left(\frac{360n}{365}\right)\right] \quad (\text{W/m}^{2})
\end{equation}

\begin{equation}
\label{beam1}
I_{B}=Ae^{-km}
\end{equation}

\begin{equation}
\label{beam2}
A=1160+75\sin\left[\frac{360}{365}(n-275)\right] \quad (\text{W/m}^{2})
\end{equation}

\begin{equation}
\label{beam3}
k=0.174+0.035\sin\left[\frac{360}{365}(n-100)\right]
\end{equation}

\begin{equation}
\label{beamc1}
I_{BC}=I_{B}\cos{(\theta)}
\end{equation}

\begin{equation}
\label{beamc2}
I_{BH}=I_{B}\cos{(90^{\circ}-\beta)}=I_{B}\sin{(\beta)}
\end{equation}

\begin{equation}
\label{beamc3}
\cos{(\theta)}=\cos{(\beta)}\cos{(\phi_{S}-\phi_{C})}\sin{(\Sigma)}+\sin{(\beta)}\cos{(\Sigma)}
\end{equation}

\begin{equation}
\label{diff1}
I_{DH}=CI_{B}
\end{equation}

\begin{equation}
\label{diff2}
C=0.095+0.04\sin\left[\frac{360}{365}(n-100)\right]
\end{equation}


\begin{equation}
\label{diffc1}
I_{DC}=I_{DH}\left(\frac{1+\cos{(\Sigma)}}{2}\right)=CI_{B}\left(\frac{1+\cos{(\Sigma)}}{2}\right)
\end{equation}

\begin{equation}
\label{ref1}
I_{RC}=\rho(I_{BH}+I_{DH})\left(\frac{1-\cos{(\Sigma)}}{2}\right)
\end{equation}

\begin{equation}
\label{ref2}
I_{RC}=\rho I_{B}(\sin{(\beta)}+C)\left(\frac{1-\cos{(\Sigma)}}{2}\right)
\end{equation}

\begin{equation}
\label{totc1}
I_{C}=I_{BC}+I_{DC}+I_{RC}
\end{equation}

\begin{equation}
\label{totc2}
    \begin{aligned}
        I_{C}
        =Ae^{-km} \left[ \cos{(\beta)}\cos{(\phi_{S}-\phi_{C})}\sin{(\Sigma)} \\
       & + \sin{(\beta)}\cos{(\Sigma)} + C\left(\frac{1+\cos{(\Sigma)}}{2}\right) \\ \\
      & +\rho(\sin{(\beta)}+C)\left.\left(\frac{1-\cos{(\Sigma)}}{2}\right)\right]
   \end{aligned}
\end{equation}

\begin{equation}
\label{tda1}
I_{BC}=I_{B}
\end{equation}

\begin{equation}
\label{tda2}
I_{DC}=CI_{B} \left[ \frac{1+\cos{(90^{\circ}-\beta)}}{2} \right]
\end{equation}

\begin{equation}
\label{tda3}
I_{RC}=\rho(I_{BH}+I_{DH})\left[\frac{1-\cos{(90^{\circ}-\beta)}}{2}\right]
\end{equation}

\begin{equation}
\label{sat1}
\Sigma_{effective}=90^{\circ}-\beta +\delta
\end{equation}

\begin{equation}
\label{sat2}
I_{BC}=I_{B}\cos{(\delta)}
\end{equation}

\begin{equation}
\label{sat3}
I_{DC}=CI_{B} \left[ \frac{1+\cos{(90^{\circ}-\beta+\delta)}}{2} \right]
\end{equation}


\begin{equation}
\label{sat4}
I_{RC}=\rho(I_{BH}+I_{DH})\left[\frac{1-\cos{(90^{\circ}-\beta+\delta)}}{2}\right]
\end{equation}




\end{document}



\\
\begin{equation}
\begin{aligned} % use \\ & for braking line

 \end{aligned}
\end{equation}

\begin{equation}
\begin{split}

\end{split
\end{equation*}

\begin{eqnarray}
& &%% use apersand around symbols to be centered
\end{eqnarray}

\begin{equation*}

\end{equation*}


\\

\end{document}

